\subparagraph{Methods}\label{methods}

MG7 is an open source tool implemented in Java and Scala, based on cloud
computing (Amazon Web Services). The graph data platform Bio4j
(www.bio4j.com) is used for retrieving taxonomy related information,
while Nispero (http://ohnosequences.com/nispero) is used for
distributing and coordinating compute tasks.

\subparagraph{Results}\label{results}

MG7 is an open-source, fast and horizontally scalable tool for community
profiling based on the analysis of 16S metagenomics data. It is entirely
cloud-based and specifically designed to take advantage of it: it
performs the community profiling of a sample starting from raw Illumina
reads in approximately 1 hour, needing approximately the same time for
doing the same on hundreds of samples, adjusting automatically the
computation capacity to the resources needed in each project. The
taxonomic assignment can be done using a Best BLAST hit paradigm or a
Lowest Common ancestor Paradigm; the user can choose between both
assignment algorithms and setting the similarity parameters required for
the assignment.

As an output, MG7 generates the frequencies of all the identified taxa
in any of the samples in tab-separated value text files as well as in
the standard BIOM format compliant with other metagenomics tools. This
output includes direct assignment frequencies and cumulative frequencies
based on the hierarchical structure of the taxonomy tree. It also
provides with output files suitable for generating heat-map
representations.

MG7 is an open-source tool available under the AGPLv3 license

This project is funded in part by the ITN FP7 project INTERCROSSING
(Grant 289974) and the Spanish CDTI (Centro para el Desarrollo
Tecnológico Industrial) grant NEXTMICRO, ref. IDI-20120242.
