
%%%%%%%%%%%%%%%%%%%%%%% file typeinst.tex %%%%%%%%%%%%%%%%%%%%%%%%%
%
% This is the LaTeX source for the instructions to authors using
% the LaTeX document class 'llncs.cls' for contributions to
% the Lecture Notes in Computer Sciences series.
% http://www.springer.com/lncs       Springer Heidelberg 2006/05/04
%
% It may be used as a template for your own input - copy it
% to a new file with a new name and use it as the basis
% for your article.
%
% NB: the document class 'llncs' has its own and detailed documentation, see
% ftp://ftp.springer.de/data/pubftp/pub/tex/latex/llncs/latex2e/llncsdoc.pdf
%
%%%%%%%%%%%%%%%%%%%%%%%%%%%%%%%%%%%%%%%%%%%%%%%%%%%%%%%%%%%%%%%%%%%


\documentclass[a4paper]{llncs}

\usepackage{amssymb}
\setcounter{tocdepth}{3}
\usepackage{graphicx}

\usepackage{url}
\urldef{\mailsa}\path|{alfred.hofmann, ursula.barth, ingrid.haas, frank.holzwarth,|
\urldef{\mailsb}\path|anna.kramer, leonie.kunz, christine.reiss, nicole.sator,|
\urldef{\mailsc}\path|erika.siebert-cole, peter.strasser, lncs}@springer.com|    
\newcommand{\keywords}[1]{\par\addvspace\baselineskip
\noindent\keywordname\enspace\ignorespaces#1}

\begin{document}

\mainmatter  % start of an individual contribution

% first the title is needed
\title{MG7: A fast horizontally scalable tool based on cloud computing and graph databases for microbial community profiling}

% a short form should be given in case it is too long for the running head
% \titlerunning{Lecture Notes in Computer Science: Authors' Instructions}

% the name(s) of the author(s) follow(s) next
%
% NB: Chinese authors should write their first names(s) in front of
% their surnames. This ensures that the names appear correctly in
% the running heads and the author index.
%
\author{Evdokim Kovach%
\and Alexey Alekhin\and Marina Manrique\and Pablo Pareja-Tobes\and\\ %
Eduardo Pareja\and Raquel Tobes\and Eduardo Pareja-Tobes%
}
%
% \authorrunning{Lecture Notes in Computer Science: Authors' Instructions}
% (feature abused for this document to repeat the title also on left hand pages)

% the affiliations are given next; don't give your e-mail address
% unless you accept that it will be published
\institute{Oh no sequences! Research Group, Era7 bioinformatics
\\
\email{eparejatobes@ohnosequences.com}
% \url{http://www.springer.com/lncs}
}

%
% NB: a more complex sample for affiliations and the mapping to the
% corresponding authors can be found in the file "llncs.dem"
% (search for the string "\mainmatter" where a contribution starts).
% "llncs.dem" accompanies the document class "llncs.cls".
%

% \toctitle{Lecture Notes in Computer Science}
% \tocauthor{Authors' Instructions}
\maketitle


\begin{abstract}

\textbf{Methods}\\

MG7 is an open source tool implemented in Java and Scala, based on cloud
computing (Amazon Web Services). The graph data platform Bio4j
(www.bio4j.com) is used for retrieving taxonomy related information,
while Nispero (http://ohnosequences.com/nispero) is used for
distributing and coordinating compute tasks.\\

\textbf{Results}\\

MG7 is an open-source, fast and horizontally scalable tool for community
profiling based on the analysis of 16S metagenomics data. It is entirely
cloud-based and specifically designed to take advantage of it: it
performs the community profiling of a sample starting from raw Illumina
reads in approximately 1 hour, needing approximately the same time for
doing the same on hundreds of samples, adjusting automatically the
computation capacity to the resources needed in each project. The
taxonomic assignment can be done using a Best BLAST hit paradigm or a
Lowest Common ancestor Paradigm; the user can choose between both
assignment algorithms and setting the similarity parameters required for
the assignment.

As an output, MG7 generates the frequencies of all the identified taxa
in any of the samples in tab-separated value text files as well as in
the standard BIOM format compliant with other metagenomics tools. This
output includes direct assignment frequencies and cumulative frequencies
based on the hierarchical structure of the taxonomy tree. It also
provides with output files suitable for generating heat-map
representations.

MG7 is an open-source tool available under the AGPLv3 license

This project is funded in part by the ITN FP7 project INTERCROSSING
(Grant 289974) and the Spanish CDTI (Centro para el Desarrollo
Tecnológico Industrial) grant NEXTMICRO, ref. IDI-20120242.
\end{abstract}

\end{document}