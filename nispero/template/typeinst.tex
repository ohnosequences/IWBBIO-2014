
%%%%%%%%%%%%%%%%%%%%%%% file typeinst.tex %%%%%%%%%%%%%%%%%%%%%%%%%
%
% This is the LaTeX source for the instructions to authors using
% the LaTeX document class 'llncs.cls' for contributions to
% the Lecture Notes in Computer Sciences series.
% http://www.springer.com/lncs       Springer Heidelberg 2006/05/04
%
% It may be used as a template for your own input - copy it
% to a new file with a new name and use it as the basis
% for your article.
%
% NB: the document class 'llncs' has its own and detailed documentation, see
% ftp://ftp.springer.de/data/pubftp/pub/tex/latex/llncs/latex2e/llncsdoc.pdf
%
%%%%%%%%%%%%%%%%%%%%%%%%%%%%%%%%%%%%%%%%%%%%%%%%%%%%%%%%%%%%%%%%%%%


\documentclass[a4paper]{llncs}

\usepackage{amssymb}
\setcounter{tocdepth}{3}
\usepackage{graphicx}

\usepackage{url}
\urldef{\mailsa}\path|{alfred.hofmann, ursula.barth, ingrid.haas, frank.holzwarth,|
\urldef{\mailsb}\path|anna.kramer, leonie.kunz, christine.reiss, nicole.sator,|
\urldef{\mailsc}\path|erika.siebert-cole, peter.strasser, lncs}@springer.com|    
\newcommand{\keywords}[1]{\par\addvspace\baselineskip
\noindent\keywordname\enspace\ignorespaces#1}

\begin{document}

\mainmatter  % start of an individual contribution

% first the title is needed
\title{Nispero: a cloud-computing based Scala tool specially suited for bioinformatics data processing}

% a short form should be given in case it is too long for the running head
% \titlerunning{Lecture Notes in Computer Science: Authors' Instructions}

% the name(s) of the author(s) follow(s) next
%
% NB: Chinese authors should write their first names(s) in front of
% their surnames. This ensures that the names appear correctly in
% the running heads and the author index.
%
\author{Evdokim Kovach%
\and Alexey Alekhin\and Marina Manrique\and Pablo Pareja-Tobes\and\\ %
Eduardo Pareja\and Raquel Tobes\and Eduardo Pareja-Tobes%
}
%
% \authorrunning{Lecture Notes in Computer Science: Authors' Instructions}
% (feature abused for this document to repeat the title also on left hand pages)

% the affiliations are given next; don't give your e-mail address
% unless you accept that it will be published
\institute{Oh no sequences! Research Group, Era7 bioinformatics
\\
\email{eparejatobes@ohnosequences.com}
% \url{http://www.springer.com/lncs}
}

%
% NB: a more complex sample for affiliations and the mapping to the
% corresponding authors can be found in the file "llncs.dem"
% (search for the string "\mainmatter" where a contribution starts).
% "llncs.dem" accompanies the document class "llncs.cls".
%

% \toctitle{Lecture Notes in Computer Science}
% \tocauthor{Authors' Instructions}
\maketitle


\begin{abstract}
Nowadays it is widely accepted that the bioinformatics data analysis is
a real bottleneck in many research activities related to life sciences.
High-throughput technologies like Next Generation Sequencing (NGS) have
completely reshaped the biology and bioinformatics landscape.
Undoubtedly NGS has allowed important progress in many life-sciences
related fields but has also presented interesting challenges in terms of
computation capabilities and algorithms. Many kinds of tasks related
with NGS data analysis, as well as other bioinformatics data analysis,
can be computed in a parallel, independent way; taking the maximum
advantage of this can obviously help in leveraging the analysis
bottleneck.

Given the way NGS data is generated scalability plays also an important
role in its analysis. NGS data is not generated in a continous fashion
but in a batch way, thus the computation needs can be dramatically
different at different points.

Cloud computing provides a perfect framework for systems with these two
requirements: parallel and scalable. Besides, it allows adjusting the
computation power on demand, and thus not being attached to (and paying
for) a fixed compute infrastructure.

Nispero is a Scala library for declaring stateless computations and
scaling them using cloud computing, in particular a combination of
services from AWS (Amazon Web Services). Some highlights are:

\begin{itemize}
\itemsep1pt\parskip0pt\parsep0pt
\item
  strongly typed configuration based on Scala code
\item
  CRDT-like semantics (a nispero instance is essentially a morphism
  between idempotent commutative monoids)
\item
  automatic deploy/undeploy
\end{itemize}

Nispero relies on the EC2 service (Elastic Compute Cloud) to carry out
the computations, on the S3 service (Simple Storage Service) for data
storage and on SQS (Simple Queue Service) and SNS (Simple Notification
Service) for communication between the different system components.

A Nispero system is composed by:

\begin{itemize}
\itemsep1pt\parskip0pt\parsep0pt
\item
  a ``console'' instance that tracks at any moment the status of the
  whole system giving the user the opportunity to check at any point the
  current status of the computations, workers, etc.
\item
  a ``manager'' instance that is in charge of deploying and undeploying
  the group of workers
\item
  a set of ``workers'' that performs the computations/tasks in a
  parallel, independent way
\item
  SQS queues for ``input'', ``output'' and ``error'' messages
\item
  S3 objects for ``input'' and ``output'' files
\end{itemize}

The lifecycle of a Nispero system is simple but robust. It starts with
the launch of the ``console'' and ``manager'' instances, the ``manager''
then takes the tasks from an S3 object, publishes them in a SQS queue
and launches the workers. The workers take the messages with the tasks
from the corresponding SQS queue (i.e.~the ``input'' queue) in an
independent, parallel way. Once they have finished the computation they
put the results of the computation in S3 objects, publish a message in
the ``output'' SQS queue and delete the input message of the
corresponding task from the ``input'' queue.

Nispero is an open-source project released under AGPLv3 license. 

The source code is available at \url{https://github.com/ohnosequences/nispero}.

This project is funded in part by the ITN FP7 project INTERCROSSING
(Grant 289974)
\end{abstract}

\end{document}